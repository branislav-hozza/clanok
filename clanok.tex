% Metódy inžinierskej práce

\documentclass[10pt,twoside,slovak,a4paper]{article}

\usepackage[slovak]{babel}
%\usepackage[T1]{fontenc}
\usepackage[IL2]{fontenc} % lepšia sadzba písmena Ľ než v T1
\usepackage[utf8]{inputenc}
\usepackage{graphicx}
\usepackage{float}
\usepackage{url} % príkaz \url na formátovanie URL
\usepackage{hyperref} % odkazy v texte budú aktívne (pri niektorých triedach dokumentov spôsobuje posun textu)

\usepackage{cite}
%\usepackage{times}

\pagestyle{headings}
\pagestyle{headings}

\title{Využitie Natural Language Processing pre lepšie učenie jazyka\thanks{Semestrálny projekt v predmete Metódy inžinierskej práce, ak. rok 2020/21, vedenie: J. Sitarčík}} 

\author{Branislav Hozza\\[2pt]
	{\small Slovenská technická univerzita v Bratislave}\\
	{\small Fakulta informatiky a informačných technológií}\\
	{\small \texttt{xhozza@stuba.sk}}
	}

\date{\small 8. október 2020} 

\begin{document}

\maketitle

\begin{abstract}
	V tejto práci sa zameriavame na zlepšenie učenia jazyka pomocou NLP \cite{NLP}. 
	Výhody NLP majú bohaté využitie v učení ako napríklad prístup k obrovskému zdroju textov. 
	NLP sa dnes využíva v mnohých odvetviach či už ako nástroje na rozpoznávanie reči alebo pomocník na opravu gramatických chýb. 
	V edukačnom systéme sa dá hlavne využiť na výučbu jazykov. NLP dokáže vyhodnocovať komplexnosť textu a kontrolovať gramatiku či plagiátorstvo. 
	Tento dokument bude zameraný hlavne na možné využitia a výzvy, ktorým musíme čeliť pri využití NLP technológie. 
	Zameriam sa na to ako NLP učí o jazykoch a ako učí samotný jazyk.
\end{abstract}

\section{Úvod}

NLP sa v dnešnej dobe používa takmer v každom odvetví, či už v zdravotníctve, v informatike, kontrola textu, dátová analýza, atď. 
V edukačnom systéme si môžeme ukázať v zozname \ref{zoznam_1}.Táto technológia má veľký potenciál vo využití pri učení anglického jazyka. 
Pri tejto forme štúdia vieme NLP využiť na kontrolu gramatiky pri písaní esejí, učenie slovíčok alebo správnej výslovnosti.
Ako najväčšiu výhodu využitia tejto technológie v praxi vidíme hodnotenie testov so skalárne veľkým počtom testovaných študentov. 
Okrem korektnosti testov sme schopní taktiež kontrolovať plagiátorstvo textu.
\begin{figure}[H]
	\centering
	\begin{itemize}\label{zoznam_1}
			\item Vyučovanie linguistických predmetov.
			– napr., čítanie, písánie, rozprávanie
			\item Používanie NLP v potrebách študentov alebo učiteľov
			– napr., knihy, učebné materiály, softvér
			\item Učenie matematiky alebo fyziky
			– napr., vytváranie slovných úloch, generovanie príkladov
			\end{itemize}
	\caption{Zoznam 1}
\end{figure}
\section{Natural Language Processing} \label{NLP}
Je to druh umelej inteligencie, zameranej na prácu s textom a obrázkami. 
NLP má za sebou 70 rokov vývoja a prvé zmienky sú z roku 1950\cite{historia}. 
Táto technológia je spojením linguistiky a informatiky, 
pričom vzniká snaha aby stroj porozumel prirodzenej reči človeka.


\section{Učenie jazyka} \label{ucenie_jazyka}

\ldots


\section{Učenie pomocout NLP} \label{ucenie_pomocou_nlp}
V článku z roku 2011\cite{clanok_o_studovani} bolo dokázané, že skupina detí ktoré sa učia pomocou živého učiteľa, pochopili látku lepšie ako žiaci,
ktorí sa učili pomocou PC.


\ldots


\section{Spracovanie jazyka} \label{spracovanie_jazyka}

\ldots


\section{Záver} \label{zaver} % prípadne iný variant názvu



\bibliography{literatura}
\bibliographystyle{plain} 
\end{document}
