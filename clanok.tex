% Metódy inžinierskej práce

\documentclass[10pt,twoside,slovak,a4paper]{article}

\usepackage[slovak]{babel}
%\usepackage[T1]{fontenc}
\usepackage[IL2]{fontenc} % lepšia sadzba písmena Ľ než v T1
\usepackage[utf8]{inputenc}
\usepackage{graphicx}
\usepackage{url} % príkaz \url na formátovanie URL
\usepackage{hyperref} % odkazy v texte budú aktívne (pri niektorých triedach dokumentov spôsobuje posun textu)

\usepackage{cite}
%\usepackage{times}

\pagestyle{headings}

\title{Využitie Natural Language Processing pre lepšie učenie jazyka\thanks{Semestrálny projekt v predmete Metódy inžinierskej práce, ak. rok 2020/21, vedenie: J. Sitarčík}} 

\author{Branislav Hozza\\[2pt]
	{\small Slovenská technická univerzita v Bratislave}\\
	{\small Fakulta informatiky a informačných technológií}\\
	{\small \texttt{xhozza@stuba.sk}}
	}

\date{\small 8. október 2020} 



\begin{document}

\maketitle

\begin{abstract}
	V tejto práci sa zameriavame na zlepšenie učenia jazyka pomocou NLP \cite{NLP}. Výhody NLP majú bohaté využitie v učení ako napríklad prístup k obrovskému zdroju textov. NLP sa dnes využíva v mnohých odvetviach či už ako nástroje na rozpoznávanie reči alebo pomocník na opravu gramatických chýb. V edukačnom systéme sa dá hlavne využiť na výučbu jazykov. NLP dokáže vyhodnocovať komplexnosť textu a kontrolovať gramatiku či plagiátorstvo. Tento dokument bude zameraný hlavne na možné využitia a výzvy, ktorým musíme čeliť pri využití NLP technológie. Zameriam sa na to ako NLP učí o jazykoch a ako učí samotný jazyk.
\end{abstract}



\section{Úvod}

NLP sa v dnešnej dobe používa takmer v každom odvetví, či už v zdravotníctve, v informatike, kontrola textu, dátová analýza, atď. V edukačnom systéme si môžeme ukázať v zozname \ref{zoznam_1}.
Táto technológia má veľký potenciál vo využití pri učení anglického jazyka. Pri tejto forme štúdia vieme NLP využiť na kontrolu gramatiky pri písaní esejí, učenie slovíčok alebo správnej výslovnosti.
Ako najväčšiu výhodu využitia tejto technológie v praxi vidíme hodnotenie testov so skalárne veľkým počtom testovaných študentov. Okrem korektnosti testov sme schopní taktiež kontrolovať plagiátorstvo textu.


\section{Nejaká časť} \label{nejaka}

Z obr.~\ref{f:rozhod} je všetko jasné. 

\begin{figure*}[tbh]
\centering
%\includegraphics[scale=1.0]{diagram.pdf}
Aj text môže byť prezentovaný ako obrázok. Stane sa z neho označný plávajúci objekt. Po vytvorení diagramu zrušte znak \texttt{\%} pred príkazom \verb|\includegraphics| označte tento riadok ako komentár (tiež pomocou znaku \texttt{\%}).
\caption{Rozhodujúci argument.}
\label{f:rozhod}
\end{figure*}



\section{Iná časť} \label{ina}

Základným problémom je teda\ldots{} Najprv sa pozrieme na nejaké vysvetlenie (časť~\ref{ina:nejake}), a potom na ešte nejaké (časť~\ref{ina:nejake}).\footnote{Niekedy môžete potrebovať aj poznámku pod čiarou.}

Môže sa zdať, že problém vlastne nejestvuje\cite{Coplien:MPD}, ale bolo dokázané, že to tak nie je~\cite{Czarnecki:Staged, Czarnecki:Progress}. Napriek tomu, aj dnes na webe narazíme na všelijaké pochybné názory\cite{PLP-Framework}. Dôležité veci možno \emph{zdôrazniť kurzívou}.


\subsection{Nejaké vysvetlenie} \label{ina:nejake}

Niekedy treba uviesť zoznam:

\begin{itemize}
\item jedna vec
\item druhá vec
	\begin{itemize}
	\item x
	\item y
	\end{itemize}
\end{itemize}

Ten istý zoznam, len číslovaný:

\begin{enumerate}
\item jedna vec
\item druhá vec
	\begin{enumerate}
	\item x
	\item y
	\end{enumerate}
\end{enumerate}


\subsection{Ešte nejaké vysvetlenie} \label{ina:este}

\paragraph{Veľmi dôležitá poznámka.}
Niekedy je potrebné nadpisom označiť odsek. Text pokračuje hneď za nadpisom.



\section{Dôležitá časť} \label{dolezita}




\section{Ešte dôležitejšia časť} \label{dolezitejsia}




\section{Záver} \label{zaver} % prípadne iný variant názvu



%\acknowledgement{Ak niekomu chcete poďakovať\ldots}


% týmto sa generuje zoznam literatúry z obsahu súboru literatura.bib podľa toho, na čo sa v článku odkazujete
\bibliography{literatura}
\bibliographystyle{plain} % prípadne alpha, abbrv alebo hociktorý iný
\end{document}
